\section{Memory Layout}
The LPC1768 provides two RAM memory regions of 32kB each.
The first region contains the \texttt{.data}-segment and the 
\texttt{.bss}-segment of non OpenPEARL-application tasks.

The second region contains the stacks and tcbs of the OpenPEARL application 
tasks and the FreeRTOS heap. This is solved via the gcc-extention
\texttt{\_\_attribute\_\_((section ("PEARL\_APPLICATION")))}
in the task definition in \texttt{GenericTask.h}.
The linker script gathers these elements into the second memory region.

This partitioning allows an easy distribution of the data into the two
RAM regions. There were no considerations about speed. The second memory
block seems to be slower in access than the first memory region.
 
\section{Principle of the Time Base}
The LPC1768 is a Cortex-M processor. FreeRTOS uses the systick timer 
for the sytem tick interrupt. 
\begin{description}
\item[FreeRTOS/addOns/CortexMclock.cc] provides a more precise value to
 the current time than on tick.
This is done
by the calulation of the fraction, which is passed in the current tick.
This calulation is based on the Cortex-M specific systick registers
{\em SYSTICK\_CURRENT\_VALUE\_REG} and {\em SYSTICK\_LOAD\_REG}.
\item[lpc1768/Lpc17xxRTC.cc] provides a system time base, which is 
initialized with the RTC value at system start.
\item[lpc1768/Lpc17xxTimer0.cc] provides a system time base, which works 
with the TIMER0.
\end{description}

\section{System Time Base Selection}
Lpc17xxClock provides a pseudo device in order
   to specify the time base for the PEARL application individually. E.g.

\begin{verbatim}
SYSTEM;
  myClock: Lpc17xxClock(1);  ! start with the time in the RTC
\end{verbatim}

The following modes are avaliable:
\begin{description}
\item[Lpc17xxClock(0)] the default time base. It starts at 1.1.1970 0:00
   and works with the systick based relative ticks with a resolution of 1 ms.
\item[Lpc17xxClock(1)] is similar to Lpc17xxClock(1), but it provides
   interpolated time values  for the \texttt{NOW}-function 
   from the registers of the systick timer.
\item[Lpc17xxClock(2)] differs from \texttt{Lpc17xxClock(1)} by the fact 
    that the initial date and time is taken from the RTC.
    In case that the RTC does not contain a useful value, the RTC
    is initialized with the default date of 1.1.2016 0:0:0.
    If the RTC does not start, the programs exits.
\item[Lpc17xxClock(3)] operates with a separate timer. This method
    provides a clock resolution of $10 \mu s$. The initial time is taken
    from the RTC like described in \texttt{Lpc17xxClock(2)}.
\item[Lpc17xxClock(4)] is similar to \texttt{Lpc17xxClock(3)}. 
    The difference is a synchronisation to the 1 seconds tick of the RTC.
    This causes small drifts in frequency but there is no gap in time 
    between resets. The synchronisation starts after 2 seconda.
    It takes approximatelly  5 seconds to enshure a time difference of
     less that $10\mu s$.
\end{description}

The preferred clocks are 
\begin{itemize}
\item \texttt{Lpc17xxClock(3)} for closed loop control systems
\item \texttt{Lpc17xxClock(4)} for open loop control systems
\end{itemize} 


\section{Power-On-Self-Test (POST)}
This module provides some information about the system usage.

Tests for e.g. memory are not implemented yet.

The POST may be selected via a pseudo device in the systempart of the 
OpenPEARL application.

\begin{description}
\item[Post(0)] invokes the POST module to display the POST status
\item[Post(1)] in addition to Post(0), the module wait for user input
   to e.q. set the RTC
\end{description}


\section{class PrioMapper}
The class PrioMapper performs a 1 by 1 mapping of PEARL priorities to FreeRTOS
priorities. For background operations, some additional priorities are reserved.

The background threads are for e.g.  the system thread or spooler threads
for i/o operations.

The priorities of these threads are stored as defines in the file
\texttt{allTaskPriorities.h}
 
