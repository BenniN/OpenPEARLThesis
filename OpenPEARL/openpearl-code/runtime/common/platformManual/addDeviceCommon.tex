\section{Common Interface for User supplied Drivers for Basic Dations
for all Platforms}
For new  system interfaces a C++ class must be supplied to 
provide the system device.

\subsection{System Dation Name}
Define the name and parameters for the system dation.
The new class must be derived from \verb|SystemDationB|.
The parameters will be passed to the constructor of the class.
The parameters of type \verb|int| and \verb|char*| are supported
by the InterModuleChecker. The IMC maps FIXED and BIT-constants 
to \verb|int| and character constants to \verb|char*|

Basic dations are used in combination with the PEARL statements TAKE and
SEND.

\subsection{Open}
A method \verb|dationOpen(int openParam=0, Character<S>* idf=0, Fixed<31>* rstValue)|
must be provided.
The method is called when the PEARL program opens the dation.
The specified open parameters are passed bitwise encoded in \verb|openParams|.
If IDF- and/or RST-value are specified they are passed as pointers, else
a null pointer is given.
The method should make shure that the dation is not opened twice and
the open parameters are ok.

\paragraph{Note:} The idf parameter is a templated type.
 Thus the dationOpen-method should be realized as a template method,
 which delegates the concrete value of the parameter to an
 internalOpen-method in an appropriate way.

\subsection{Close}
A method \verb|dationClose(int closeParams=0, Fixed<31>* rstValue)|
must be provided.
The method is called when the PEARL program closes the dation.
The specified close parameters are passed bitwise encoded in \verb|closeParams|.
If an RST-value is specified it passed as pointer, else
a null pointer is given.
The method should make shure that the dation is not closed twice and
the close parameters are ok.

\subsection{Read}
A method \verb|dationRead(void * destination, size_t size)|
must be provided. The method shall transfer \verb|size| bytes of
input data to the location specified by \verb|destination|.
The runtime system expects that this method blocks the calling thread,
if no data are available.
The method should make shure that only an opened dation may be used.
If the device requires blocking of the calling PEARL task, it 
must use a suitable blocking mechanism. In the linux environment,
the linux device driver does this implicitly. 
The other environments like FreeRTOS a semaphore should be used.


\subsection{Write}
A method \verb|dationWrite(void * source, size_t size)|
must be provided. The method shall transfer \verb|size| bytes of
output data from the location specified by \verb|source|.
The runtime system expects that this method blocks the calling thread,
until all data is transfered.
The method should make shure that only an opened dation may be used.

\subsection{Error Conditions}
The methods may throw C++ exceptions, which are defines in the signal list.
Existing signals should be reused as far as possible.

\section{Common Interface for User supplied Drivers for Non Basic Dations}
For currently not supported system interfaces a C++ class must be supplied to 
provide the system device.

\subsection{System Dation Name}
Define the name and parameters for the system dation.
The new class must be derived from \verb|SystemDationNB|.
The parameters will be passed to the constructor of the class.

Non basic dations are used in combination with the PEARL statements READ, WRITE
or PUT,GET. 

\subsection{Capabilities}
The input and output routines for non basic dations 
need information about the capabilities of the
real device. The method \verb|capabilities()| must return a set of
capabilities, which are provided by the device. The capabilities are
defines as \verb|enum| in the \verb|Dation|-class.

The runtime system checks whether the requested operations fit to the 
device capabilities. The capabilities of the device are defined by this
method. Multiple capabilities from the list below may be set via bitwise
or operations.

\begin{description}
\item[IN] the device may be used as pure input dation
\item[OUT] the device may be used as pure output dation
\item[INOUT] the device may be used as dation for random directions
    input and output
\item[IDF] the device {\em needs} a file name for the open-statement
\item[OLD] the device may contain persistent files,
          which are indented to be read
\item[NEW] the device allows the creation of files 
\item[ANY] the device  supports files at all; at least 
            this attribut should be set if IDF is set
\item[PRM] the device supports persistent files
\item[CAN] the device supports the removal of files
\item[DIRECT] the device supports direct positioning (on given byte adresses)
\item[FORWARD] the device allows sequential read and write operations
\item[FORBACK] {\em NOT SUPPORTED} --- the device allows relative positioning
\end{description}
Note that the capability
\begin{itemize}
\item IDF needs any of the capabilites OLD, NEW, ANY.
\item OLD needs the capability IN
\item NEW needs the capability OUT
\item FORWARD and DIRECT may be both
\end{itemize}

\subsection{Open}
A method \verb|dationOpen(const char * idfValue, int openParams)|
must be provided.
The method is called when the PEARL program opens the dation.
The current {\em openParams} must be treated according the language definition.

\subsection{Close}
A method \verb|dationClose(int closeParams)|
must be provided.
The method is called when the PEARL program closes the dation.
The current {\em closeParams} must be treated according the language 
definition.

\subsection{Read}
A method \verb|dationRead(void * destination, size_t size)|
must be provided. The method shall transfer \verb|size| bytes of
input data to the location specified by \verb|destination|.
The runtime system expects that this method blocks the calling thread,
if no data are available. 

\subsection{Write}
A method \verb|dationWrite(void * source, size_t size)|
must be provided. The method shall transfer \verb|size| bytes of
output data from the location specified by \verb|source|.
The runtime system expects that this method blocks the calling thread,
until all data is transfered.

\subsection{Error Conditions}
The methods may throw C++ exceptions, which are defined in the signal list.
Existing signals should be reused as far as possible.

\section{Common Interface for Connection Providers}
OpenPEARL allows the usage of connection of system elements  with other 
system elements. These connections have a definite type which is defined
by the connection provider.
The connection interface should be platform independent in order to reuse
the connection clients without any changes. 
There is no restriction about the interface of a connection provider.  

\begin{description}
\item [required methods] are not defined, since each type of cennection
   may be different. 
    The connection provider must provide the methods, which allow the the
    clients to do their job. 
\item [setup association] between client  and provider is done by the
    IMC. The reference to the provider object is passed as first argument
    to the client object. If the provider must know the client(s),
    the creater of the client classes is responsible to register the
    client object at the provider object.
\end{description}

\paragraph{Notes:}
\begin{itemize}
\item The ctor may used parameters of type \verb|int| and \verb|char*|.
  The InterModuleChecker (IMC) maps PEARL FIXED and BIT-Values to \verb|int|
  and CHAR(..) constants to \verb|char*|.
  If other types are required, the IMC must be extended at this point.
\item A connection provider is instanciated before the main function
  is entered. In case of trouble, the ctor should exit by throwing a 
  PEARL signal and writing a log-message using the Log::error()  mathod.
\item the implentation of a connection provider must be aware, that 
   the multiple clients may interact with the provider simulaneously.
   A mutex-locking mechanism should be used to avoid race conditions.
\end{itemize}

