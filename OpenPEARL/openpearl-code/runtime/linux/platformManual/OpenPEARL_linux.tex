\documentclass[oneside,10pt]{scrbook}
\usepackage{geometry}
\usepackage{graphicx}
\usepackage{import}
\usepackage{xcolor}
\usepackage{listings}
\usepackage{caption}
\usepackage{hyperref}
\usepackage{tikz}
\usetikzlibrary{calc,positioning,shapes.geometric}

\geometry{a4paper}
\setlength{\parindent}{0pt}

%\DeclareCaptionFont{white}{\color{white}}
%DeclareCaptionFormat{listing}{\colorbox[cmyk]{0.43, 0.35, 0.35,0.01}
\DeclareCaptionFormat{listing}{\colorbox{black!15}
  {\parbox{5cm}{#1#2#3}}}
\captionsetup[lstlisting]{format=listing,
%labelfont=white,textfont=white, 
singlelinecheck=false, margin=0pt, font={bf,footnotesize}}

\begin{document}
\title{OpenPEARL - Platform Manual for Linux Systems}
\author{R. M\"uller}

\lstnewenvironment{CppCode} {
    \lstset{numbers=left,
            title={C++},
            frame=tlrb,
	    breaklines = true,
	    belowcaptionskip=4pt
    }
}%
{}%

\lstnewenvironment{PEARLCode} {
    \lstset{numbers=left,
            title={PEARL},
            frame=tlrb,
	    breaklines = true,
	    belowcaptionskip=4pt
    }
}%
{}%

\pagestyle{myheadings}
\markboth {OpenPEARL Platform Manual Linux}
          {OpenPEARL Platform Manual Linux}

\maketitle

\tableofcontents

%\input{log.tex}
%\input{interrupt.tex}
%\input{inOutput.tex}

\chapter{Installation}
The installation must be done via the git repository of the OpenPEARL project.
There are several software packages required on your linux PC. 
The repository contains an install script (\verb|installPackages|),
which identifies your
linux distribution by content of the file \verb|/etc/os-release|.

For details of the additional packages study the installation script.

\paragraph{Prerequisites}
\begin{itemize}
\item you have access to your linux system as user and root
\item you have an internet connection
\item you have a git client installed
\end{itemize}

\paragraph{Perform the following steps:}
\begin{enumerate}
\item login as normal user
\item create a working directory for the installation. \\
   e.g. /home/userx/OpenPEARL\\
    \verb|mkdir /home/userx/OpenPEARL|
\item set you current working directory to this point \\
    \verb|cd /home/userx/OpenPEARL|
\item obtain a read only copy the repository\\
    \verb|git clone git://git.code.sf.net/p/openpearl/code .|\\
    regard the point at the end of the command
\item set the working directory to the top of the real content
     in your working copy 
     (we are sorry about this unnecessary directory level): \\
    \verb|cd openpearl-code|
\item get administration priviledges\\
    \verb|sudo| or \verb|su|
\item run the script:\\
    \verb|./installPackages|\\
    and wait for completion of the installation.
    This takes about 6 minutes at an RaspberryPi 3B
\item still as administrator, execute\\
    \verb|make defconfig|\\
    \verb|make prepare|\\
    \verb|make install|\\
    this takes about 20 minutes at a Rspberry Pi 3B
\end{enumerate}

\paragraph{Verify Installation:}
Work as normal user.


\begin{enumerate}
\item \verb|cd /home/userx/OpenPEARL/openpearl-code/demos|
\item build some of the provided demonstration programs\\
   \verb|prl <demo>.prl|\\
   where \verb|<demo>| is Hello or sched\_demo
\item run the demonstration programm with \\
   \verb|./Hello| or \verb|./sched_demo|.\\
   The programs should print some messages and terminate themself.
\end{enumerate}


\chapter{Configuration}
\section{Available Options}
\begin{itemize}
\item Target platform: standard linux PC or Raspberry Pi
\item additional device support like PEAK CAN-adapter, I2C-devices,...
\end{itemize}
For details see the {\em HELP} pages in the \verb|make menuconfig| 
entries.

\section{Rebuild of the OpenPEARL Environment}
The tuning the installation must be done as administrator:
\begin{enumerate}
\item \verb|cd /home/userx/OpenPEARL/openpearl-code|
\item change the option upon your demands:\\
      \verb|make menuconfig| 
\item \verb|make install|
\end{enumerate}

\section{Configuration Option for the individual Application}
\import{../../common/platformManual/}{configurationCommon.tex}
\import{./}{log.tex}
\import{./}{pearlrc.tex}

\chapter{Release Information}
\section{Supported Language Elements}
The OpenPEARL project aims to support all language elements  as
described in the language report.

This version does not support:\footnote{The numbers in braces refer the ticket numbers.}

\begin{itemize}
\item STRUCT
\item TYPE
\item no SIGNAL treatment (\#127)
%%% \item no INTERRUPT support (\#122)
%%% \item operator SIZEOF not supported (\#109)
%\item length of float constants must be clearified and is subject of 
%		change(\#120)
\item REF CHAR
\item REF used upon inititialized references does not throw a signal (\#217)
\item REF on TASK,DATION,SIGNAL,INTERRUPT not tested (\#196,\#197)
\item no support for multi modules (\#193)
\item I/O-related
   \begin{itemize}
   %\item no warning if multiple positions formats are inside a READ/WRITE statement (\#126)
   %\item E-format does not work properly (\#210)
   %\item CONVERT not supported (\#215)
   %\item additional X-elements in some PUT statements (\#221)
   %\item format elements PAGE,ADV,LINE,COL,POS,SOP,EOF not supported(\#216,\#222)
   \item CYCLIC only available on system dations of type DIRECT (\#226)
   \item repetition factor in i/o formats and variable array slices 
      does not work in i/o statements(\#232)
   \item TFU not supported (\#232,\#237)
%   \item loop index not accessible in i/o statements (\#239)
%   \item type of dation element in i/o statements not checked (\#251)
   \item no array slices as i/o element are supported (\#223,\#252)
   \item no R-Format (\#254)
   \end{itemize}
%\item no constants expressions treated by the compiler (\#248)
\item BIT-slices with size 1 do not work with variable slice index (\#243)
\item BIT and CHAR-slices not allowed on left hand side and in input statements
    (\#224)
%\item control characters in string constants does not work if 
%    there is a newline in the control character sequence(\#203) 

\item Semantic analysis for good diagnostic error messages.
     In case of problems error messages from the c++ compiler may appear. 
\item Semantic analysis for program quality analysis (e.g. unused variables)
\end{itemize}

\section{Restrictions for privileged and unprivileded users}
\begin{itemize}
\item The priority based scheduler of the linux system is not
   accessible for normal users. Thus application will operate for 
   normal users with
   the usual completely fair scheduler. The priority values are 
   not regarded in the scheduling.
   It is possible by system configuation to allow normal users
   the access to real-time priorities. Just modify the file
   \verb|/etc/security/limits.conf| by a line like\newline
   \verb|*  - rtprio 99|\newline
   This allows all users to use real-time priorities.
   \newline
   If you work via remote desktop protocol you must force pam to
   recognize the limits.conf by adding the next line to \verb|/etc/pam.d/common-session|
   \newline
   \verb|session    required   pam_limits.so|
\item If the application is run with root priviledges, the socalled 
  round-robin scheduler is used. This provides only 100 priority values.
  For administrative reasons, some priorities are clipped by the OpenPEARL
  system.
  The remaining priorities are: 1-97 and 255
\end{itemize}

\section{Tuning Information}
\begin{itemize}
\item The current version uses only one processor core.
     There is an option to use other cores but this is not tested yet.
\item There is no memory locking used.
     This would increase the execution speed but effect the overall
     behavior of the linux machine.
\item The optimizer switches of the c-compiler are not optimized.
   The debugging option is still set!
\item The stack size for the tasks may be controlled with the
   \texttt{ulimit -s xx} command of the command shell (bash).
\end{itemize}

\iffalse
\section{Execution Times}
The execution speed of some operations are derived from the system clock.
The resolution of the system clock depend on the hardware and the 
presence of other processes.
The time stamps were taken with the \texttt{NOW} builtin function.

During the measurement, no network connection was established and no
other tasks are active. The usual background activity of the linux
systems was present.

For details about the measurements please consult the corresponding
source code. The benchmark are located in \verb|demos/benchmark/|

\subsection{Execution Timing on Raspberry Pi 3b}

\subsection{Execution Timing on a Desktop Linux System}
The excecution speed depends on many hardware parameters. 
The following data are taken from my development machine, which consists of
Intel(R) Xeon(R) CPU E31270 @ 3.40GHz.


\begin{tabular}{|p{5cm}|l|r|}
\hline
Parameter & Test Program & Result \\
\hline
Task Activation & Tasking.prl & $59-277 \mu s$\\
\hline
Semaphore Operations & Tasking.prl &  \\
REQUEST-RELEASE non blocking & & $2 \mu s$\\
RELEASE-REQUEST with context switch & & $8-18\mu s$\\

\hline
Simple Type Operations & CharTests.prl & \\
                       & BitTests.prl & \\
\hline
Matrix Multiplication & MatMult.prl & ...\\
\hline
\end{tabular}

\paragraph{Note:} These times are not realistic, since the complete test fits in the cache of 
the processor. The development of realistic test programs 
\fi

\chapter{Usage of the OpenPEARL system}

The principle of operation is very similar to the usage of the \texttt{gcc}.

\begin{enumerate}
\item create source file
\item compile and link
\item run the application
\end{enumerate}

In order to explain the usage, we start with a copy of the hello.prl demonstration file.

\begin{enumerate}
\item copy \texttt{hello.prl} into your working directory (e.g. \texttt{/home/userx/prl}) \newline
   \texttt{cp /home/userx/OpenPEARL/openpearl/demos/hello.prl /home/userx/prl}
\item set the working directory to \newline
   \texttt{cd /home/userx/prl}
\item build the hello world application \newline
   \texttt{prl hello.prl}
\item run the hello world application \newline
   \texttt{./hello}
\end{enumerate}

For your own applications, just modify the source code of demo.prl according your demands.
You should not use \texttt{system.prl} as source file - this is a known problem of the current
build system.


\paragraph{Note:}
Please ignore the intermediate files like:
hello.cc, hello.log, hello.xml, system.cc



\chapter{Supported Devices}
\import{../../common/platformManual/}{devicesCommon.tex}
\import{./}{devices.tex}

\chapter{Supported Interrupt Sources}
\import{../../common/platformManual/}{interruptCommon.tex}
\import{./}{interrupts.tex}

\chapter{Available PEARL Signals}
The OpenPEARL signals are organized in some groups.

\import{../../common/}{signallist.tex}


\chapter{How-To: Add new System Elements}
\import{../../common/platformManual/}{addDeviceCommon.tex}
\import{./}{addDevice.tex}
\import{../../common/platformManual/}{addInterruptCommon.tex}
\import{./}{addInterrupt.tex}
\import{../../common/platformManual/}{addSignalCommon.tex}
\end{document}

