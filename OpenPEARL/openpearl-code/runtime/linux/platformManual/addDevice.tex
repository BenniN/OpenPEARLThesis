\section{Linux specific Details in Device Driver Implementation}
In linux the system devices are represented by a device driver of
the linux kernel.
It is possible to access the devices via the normal interface
with {\em open}, {\em close}, {\em read}, {\em write} and {\em ioctl}.
The device node name in the folder \verb|/dev| must have access
priviledges for {\em users}.

These functions should be used in the methods of the C++ class 
for the device driver.

\subsection{First Considerations}
Some considerations must be done at the early beginning:
\begin{itemize}
\item Which i/o perations are intended?\newline
In case of  READ/WRITE or PUT/GET the parent class is SystemDationNB or 
InterruptableSystemDationNB.
In case of  TAKE/SENDE the parent class is SystemDationB or 
InterruptableSystemDationB.
\item Are there blocking operation in the device code?
\newline
Use the InterruptableSystemDationB or InterruptableSystemDationNB as parent
class and treat the return code EINTR as shown in the following example.
\item Provides the device a service to other system devices eg. like
an I2C-bus?
Define an interface for the client devices in a separate class.
\end{itemize}

\subsection{Interruption of system calls}
Each user dation provides a mutex mechanism. 
Some device drivers need a mutex-mechanism to enable atomic behavior
when used from different tasks.
This is obviously true for connection providers.

The read/write system calls will be interrupted by the UNIX signal
\verb|SIG_CANCEL_IO|. This leed to an error value \texttt{EINTR} for
the system call or the according value of \texttt{errno} for fread/fwrite.

In case of a SUSPEND request, the system call must be repeated after a 
CONTINUE opreation. Locked ressources remein locked.
In case of a TERMIANTE request, alle locks in the i/o stack must become
released. 
Both cases are treated in the class InterruptableSystemDationNB and 
InterruptableSystemDationB.


\subsection{Sample code from StdIn.cc}
\begin{verbatim}
#include "Task.h"

   void StdIn::dationRead(void * destination, size_t size) {
      int ret;
      int errnoCopy;

      mutex.lock();
      clearerr(fp);
      errno = 0;

      // perform the read() inside a try-catch block
      // treatCancelIO will throw an expection if the current
      // task should become terminated. In the catch block,
      // the mutex becomes released
      try {
         do {
            ret = fread(destination, size, 1, fp);

            // safe the value of errno for further evaluation
            errnoCopy = errno;

            if (ret < 1 ) {
               if (errnoCopy == EINTR) {
                  Task::currentTask()->treatCancelIO();
                  Log::info("StdIn: treatCancelIO finished");
               } else if (feof(fp)) {
                  Log::error("StdIn: error across EOF");
                  mutex.unlock();
                  throw theDationEOFSignal;
               } else {
                  // other read errors
                  Log::error("StdIn: error at read (%s)", strerror(errnoCopy));
                  mutex.unlock();
                  throw theReadingFailedSignal;
               }
            }
         } while (ret <= 0);
      } catch (TerminateRequestSignal s) {
         mutex.unlock();
         throw;
      }
      mutex.unlock();
   }


   void StdIn::dationWrite(void * source, size_t size) {
      Log::error("StdIn: write is not supported");
      throw theDationNotSupportedSignal;
   }
\end{verbatim}


\subsection{Checklist: How to create a new System Device}
In linux the system devices are represented by a device driver of
the linux kernel.
It is possible to access the devices via the normal interface
with {\em open}, {\em close}, {\em read}, {\em write} and {\em ioctl}.
The device node name in the folder \verb|/dev| must have access
priviledges for {\em users}.

These functions should be used in the methods of the C++ class 
for the device driver.


The creation of a new device is very simple:
\begin{enumerate}
\item find a suitable device name
\item pass the parameters needed for construction of the device as
      parameters
\item add a target rule for the device in the Makefile
\item create a C++ class with the name of the device and add this class
      in the Makefile
\item derive the C++ class from SystemDationB or SystemDationNB depending
      on the dation type. SystemDationB is the parent for process dations.
      SystemDationNB is parent for dations for formatted and unformatted
      i/o with PUT,GET,READ and WRITE
\item if there are blocking statements in den device code use 
      InterruptableSystemDationB or InterruptableSystemDationNB as
      parent class.
\item provide {\em capabilities} in the ctor in that way that the allowed
      capabilities are listed
\item if more than on dation may be opened in a concurrent way on the new
      device, you should provide a pool mechanisme like in Disc.
      If the new device allows only one opened dation at one time, you 
      should omit the pool.
\item provide the dationOpen, dationClose, dationRead, dationWrite 
      and dationSeek methods. Check the actual given parameters on
      validity and create Log-entries and throw exceptions in case
      of bad parameters.
      Note that dationSeek is not required, if only FORWARD dations
      may be created.
\item create a suitable XML description file, which describes the
      device characteristics.
\item If the device depends on special libraries or 
      system setup, insert a configuration item  
      in the system configuration menu (make menuconfig)
      and set the default to {\em not available} ('n').
\item modify the corresponding Makefile and add the device depending on
      the configuration item. If additional libraries are needed,
      add them to \texttt{LDPATH} variable. 
\item create unit tests for google tests framework in tests/ and add the
      tests in tests/Makefile
\item run the tests and check for proper operation of the new device
\item add the documentation of the new device to this document
\end{enumerate}

