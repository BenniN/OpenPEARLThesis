\chapter{Language Elements to be treated by the Compiler}
The following language elements must be treated by the compiler 
without suport by the runtime system.

\begin{description}
\item[LENGTH] of FIXED, FLOAT, BIT, CHAR must be evaluated by the 
    compiler. The runtime system expects concrete lengths for each element
\item[REF] corresponds to the definition of a pointer to the 
    element. The compiler should should instanciate an object
    the templated Ref-class. This class provides the operator* with
    a check, wether the referene ist not NIL.
    Note that REF CHARACTER works different. This type is supported by the
    runtime system.
\item[CONT] Follow a pointer (\verb|*| works on the Ref-object). 
		Only REF CHAR is supplied by the runtime system
\item[TYPE] is a type declaration. The runtime expects the primitive types.
   Perhaps the TYPE can be mapped on \verb|typedef|.
\item[INV] denotes invariant. This should be checked by the compiler internally.
    Perhaps the C++ \verb|const| may work. 
%\item[ARRAY] defines an array of elements (\verb|[]| should work)
\item[STRUCT] must be mapped to C++ \verb|struct| by the compiler 
(see section \ref{sec:struct})
\item[TYPE] defines a type  upon e.g. a struct (\verb|typedef| should work)
\item[DCL] defines a element. The compiler must instantiate an object
    of the requested type. The C++ storage class modifiers should work
    without problems.
\item[SPC] specifies an element, which is declared in another module as
  GLOBAL. The C++ \verb|extern| must be used.
\end{description}

 

