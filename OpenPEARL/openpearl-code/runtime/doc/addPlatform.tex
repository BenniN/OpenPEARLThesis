\section{Basic about the Build System}
The build system is based in make.
At top level of OpenPEARL the configuration is set using Kconfig.

If a new platform should be added some changes must be done to integrate
the new platform to the OpenPEARL build system.

\paragraph{Note:}
Be aware that OpenPEARL is provided under a 3-clause BSD license.
Do not integrate any parts which violates this license.

\begin{enumerate}
\item Invent a suitable identifier for the new platform
like \verb|esp32|.
\item define a numeric id for the platform like 3. This number is required
  to select a specific platform for the application build.
  Refer to the used numbers in the target 'install' 
  in \verb|runtime/Makefile|.
\item define a preprocessor identifier for small platform specific
   changes in common files for similar platforms, like \verb|OPENPEARL_ESP32|.
   Pass this identifier with the \verb|-D| option to the c/c++ compiler
\end{enumerate}

\section{Kconfig}
Add in \verb|configuration/fm| a platform specific Kconfig file and add this 
file to \verb|configuration/fm/runtime.fm|.

\begin{itemize}
\item provide an overall option \verb|XXX| 
   to enable the platform and set the default 
      to 'n'
\item provide additional options required and set then as 
   \verb|depends on XXX|
\end{itemize}

\section{Integration to the Build system}
\begin{itemize}
\item create a directory for your platform in \verb|.../runtime/| (eg. esp32)
\item provide a \verb|Makefile| in the platform directory for the
   targets
  \begin{description}
  \item[clean] as usual 
  \item[all] just build the platform specific files. Additionally,
     the build system expects
     \begin{description}
        \item[pearl.h] with a single header file for this platform with 
         all OpenPEARL specific classes. You create this with the 
         \em{include composer}
       \item[cc\_bin.inc] must invoke the compiler with all required 
                flags. Use \verb|$fn.cc| for the name of the application
                source file and \verb|system.cc| for the representation
                of the system part.
       \item[run\_bin.inc] do all operations to start the application on
               your platform. This includes specific file translations and
               downloading.
     \end{description}
  \item[install] copy the platform specific files to 
	\verb|$CONFIG\_INSTALL\_Target|
  \item[prepare] create dummy files for the files to be installed later.
    And set rw-permisions for regular users. This build is executed 
    initially with root-priviledges
  \end{description}
\item add your platform to \verb|runtime/Makefile| in the make targets
   \begin{itemize}
   \item prepare
   \item install
   \item cc\_bin.inc
   \item run\_bin.inc
   \item and a plain target for your platform, which invokes for the target
      'all' (eg. esp32)
   \end{itemize}
\item if you need special OpenPEARL signals in your platform, refer to the 
   chapter \ref{sec_signal_definition}.
\item provide a platform manual, which describes the usage of the new platform.
\end{itemize}

