\chapter{Task}
\label{task}
A task implementation is very specific for each platform.

The language report indicates the task state diagram from 
fig. \ref{taskStatesPEARL90}. There is no difference made between {\em running} and {\em runable}. The switching between these two states is done by the
operating system automatically --- not visible to the PEARL application.

\begin{figure}[bpht]
\includegraphics[width=14cm]{taskStatesPEARL.jpg}
\caption{Task statediagram derived from the language report.
The bolt operations are omitted, sice they are identical to the
semaphore operations}
\label{taskStatesPEARL90}
\end{figure}


RTOS-UH used a different model, where termination while an active i/o statement
is delayed until the next end-of-record. 
 

OpenPEARL uses an implementation which is more close to the language report.
All tasking statements which are executed by the task itself are executed 
obviously synchronously.
Tasking statements concerning other tasks are a little 
bit more difficult, because the runtime system uses internally semaphore
 and mutex
variables to avoid race conditions on i/o operations.
During such a critical region,
no task termination must occur. So the  {\em remote suspend request} and 
{\em remote termination request} just set a flag and wait for the fulfilling
 of the 
tasking statement from the adressed task. The priority of the 
task to be suspended or terminated is set to the best priority in the system.
Each task has synchronisation points,
which are
\begin{itemize}
\item  after each source statement and
\item  in each synchronisation statement and 
\item in each i/o operation on the system device.
\end{itemize}

During i/o-operation it is a little bit more complicated. OpenPEARL
recommends like stated in the language report that suspend and terminate
effect the task state immediatelly.
In order to interrupt a blocked i/o operation, each platform MUST implement 
methods in the system dation's implementation to SUSPEND and TERMINATE a task while 
performing i/o operations.

The system device driver must invoke a platform dependent method
\verb|treatCancelIO| which will throw ether an exception
(\verb|TerminateRequestSignal|) in case of a termination request
 or just suspends itself.
The TerminateRequestSignal is caught and rethrown at each block which 
locks internal mutexes or semaphores. These variables are unlocked in
the exception handler and the exception is rethrown up to the upmost i/o-API
function.

For the list of specialized tasking methods please refer the
{\em pure virtual} methods of \verb|TaskCommon| in the Doxygen
part of the documentation. The implementation on linux systems is described 
in sections \ref{linux_suspend}, \ref{linux_terminate} and for FreeRTOS based systems 
in section \ref{freertos_tasking}.


The new state diagram is shown in 
fig. \ref{taskStatesOpenPEARL}.
The transitions from the  states {\em terminatePending} and 
{\em suspendPending}  are triggered from the synchronisation points.

\begin{figure}[bpht]
\includegraphics[width=14cm]{taskStatesOpenPEARL.jpg}
\caption{Task statediagram in OpenPEARL.
The states suspendPending and terminatePending were introduced.}
\label{taskStatesOpenPEARL}
\end{figure}
To trigger the events in the state diagram suitable methods in the task 
objects are provided. 

The blocked state is treated differently depending of the block reason, which
is ether:
\begin{description}
\item[REQUEST, ENTER or RESERVE], which means that the task is waiting for 
   the synchronisation variable
\item[IO], which means that the task is currently doing an i/o-operation
   on an user dation
\item[IOWaitQueue], which means that the task is waiting in a priority based
   queue to start it's i/o-operation
\item[IO\_MULT\_IO], which means that the task is waiting in a 
   queue to start it's i/o-operation. The queue is treated by the 
   system dation like in the Console device.
\end{description}

Several classes support the tasking operations:
\begin{description}
\item[class Interrupt] provides the semantics of the PEARL language interrupts
\item[class Semaphore] provides PEARL semaphore semantics with simultanous
   locking on semaphore arrays.
\item[class TaskTimerCommon] provides an interface for the platform
   specific class \verb|TaskTimer|. 
\item[class TaskCommon] provides most of the error checking depending on
   the task state and tasking operation, as well as platform independent
   parts of the tasking operations. Theses operations delegate the concrete
   platform specific operation to the platform specific implementation
   in the class \verb|Task|. The api methods in TaskCommon perform some
    parameter checks and look the mutex for all tasks before delegating 
    the job to the platform specific implementation. 
    The mutex must be unlocked by the target specific implementation 
    after completion of the job. Some operation may need to release and 
    rerequest the mutex in the target specific job --- e.g. suspend,
    which has to block the task until it is continued by another task
    or a scheduded continue.

   Scheduled operations are possible in combination of ACTIVATE, 
   CONTINUE and RESUME.
   The schedule (like AT, ALL, AFTER, WHEN,..) is passed to the 
   tasks methods. The way of treatment (by timers, specific timer threads, ... )
   is decided by the platform code.

   For details about the API of the provided methods, please check the doxygen 
   version of the documentation.
   Please run  the command \verb|make doc| in the base dirtectory of the
   desired platform.

\item[class TaskList] provides a list of all defined PEARL tasks
\item[class TaskMonitor] keeps track of all scheduled and active tasks and
   detects the end of the application.
\item[class Control] allows setting the exit status for the application. 
  The treatment of this status is platform specific.
\end{description}

\section{Thread Safety}
The task state is stored in the task control block, which is realized
by attributes of the \verb|Task|-objects.
The mutual exclusion is realized by  public class methods
\begin{itemize}
\item \verb|TaskCommon::mutexLock()|
\item \verb|TaskCommon::mutexUnlock()|
\end{itemize}
 
This mutex is used in the task related classes, as well as in the interrupt,
semaphore, bolt and i/o related classes. 

All tasking related methods, which check or modify  the tasks state lock the 
mutex variable. Some parts of execution are delegated to platform specific
code. For details about unlocking the mutex please check the code
documentation from doxygen.



\section{Specification and Declaration}
\begin{description}
\item[SPCTASK] makes the forward declararion of the task object.
    The compiler must add the forward declaraction of all tasks
    used in the module before the first declaration, since C++
    variables have a scope from the line of declaration until 
    the end of the block, which contains the declaration.
    The macro has the tasks name as parameter.
\item[DCLTASK] defines the task itself. The DCLTASK macro is followed
    by a block of statements, which define the tasks content.
    The macro has the tasks name, priority and main-attribute as
    parameter. 
    Priority and isMain-attribute are wrapped by the compiler 
    into \verb|Prio|- and \verb|BitString<1>|-type.
\end{description}

The implementation of task handling is platform specific.
Each platform specific implementation need:
\begin{enumerate}
\item  a class {\em Task} which is derived from {\em TaskCommon}.
\item the suitable definition of the \verb|DCLTASK|- and \verb|SPCTASK|-macros
\item the implementation of the tasking methods
\item a mechanism which activates on system start
   all defined tasks containing the isMain-attribute with true. 
   The sequence is defined by the tasks priority. 
   No priority corresponds to the weakest priority.
\item a mechanism which enables a default signal handler for each task
\item a mechanism which enshures the task termination at the end of
   the task body
\end{enumerate}

\paragraph{Notes}\ \\
\begin{itemize}
\item The macros SPCTASK and DCLTASK are currently sufficient only for
    PEARL programs with one module. When multi-module programs
    are supported by the compiler, the GLOBAL-attribute should be 
    passed additionally to both macros.
\end{itemize}

\section{me-Pointer}
The DCLTASK-macro must define an object with the name of the
task with the type {\em Task}.
The macro must enshure that a variable  {\em Task * me;} exists
and points to the tasks object. This pointer is used
by the compiler to trigger tasking operations of the current
task.

This pointer is passed to every procedure as hidden (first) parameter.
By this mean, each procedure may do tasking operations upon the
task, which uses the procedure.
The {\em me}-pointer is defined to be from the base class  {\em TaskCommon}. 
The runtime tasks are derived from this class, so the pointer to the current 
task fits to {\em me}.

\section{Task Priority}
The tasks default priority is defined in the TASK-statement and passed
as second parameter in the DCLTASK-macro.
Scheduling calls may alter the tasks priority. This new priority is valid 
until a new priority is assigne. At restart of the task the default 
priority is used again. Each platform must provide a class \verb|PrioMapper|.

If the concrete target system (like linux) does not support 255 
 application priorities, the method \verb|fromPearl()| must throw
an exception.


\section{Error Tracing}
The error handling of PEARL is done with SIGNALs.
They are very similar to the exceptions in C++ and JAVA. 
The run time systems throws exepctions which are organized in a
hierarchy to catch several exceptions of one kind in one step.
To provide a backtrace to the originating PEARL source line, the compiler 
must add tracing functions.
\begin{CppCode}
Task::setLocation(const char * fileName, int lineNumber);
\end{CppCode}
The method {\em setLocation()} is supplied with the values from
the PEARL source file.

\begin{CppCode}
const char* filename =  "api.prl";
SPCTASK(t1);
SPCTASK(t2);

DCLTASK(t1,10,1) {
   me->setLocation(10,filename);
   pearlrt::Duration d(5.0);
   me->setLocation(11,filename);
   pearlrt::Duration dur;
   me->setLocation(12,filename);
   dur = d*10.0;
   me->setLocation(13,filename);
   printf("5 sec delay\n");
   me->setLocation(14,filename);
   me->resume(pearlrt::Task::AFTER, pearlrt::Clock(),d);
   me->setLocation(15,filename);    
   printf("after 5 sec all 2 sec during 30 sec activate  t2\n");
   printf("    -> 16 activations\n");
   me->setLocation(16,filename);
   t2.activate(me, pearlrt::Prio(30),
               pearlrt::Task::AFTER|pearlrt::Task::DURING|pearlrt::Task::ALL,
               pearlrt::Clock(), d, pearlrt::Duration(2.0), // at, all
               pearlrt::Clock(),d*6.0,0) ;    
}

int t2counter=0;
DCLTASK(t2,20,0) {
   me->setLocation(17,filename);
   printf("t2 started (%d)\n", ++t2counter);
   me->setLocation(18,filename);
   pearlrt::Duration d(10.0);
   me->setLocation(19,filename);
   d = d / (t2counter-2);
}
\end{CppCode}

As result, uncaught SIGNALS would produce an error message like:
\begin{verbatim}
***************************
* Signal "fixed overflow" occured in api.prl at line 19
* Task T2 terminated
***************************
\end{verbatim}

Details about signal-handling is described in the section about signals.

