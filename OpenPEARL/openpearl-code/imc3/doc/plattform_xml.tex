\chapter{Platform Specific Names}

\section{Introduction}
Each specific platform may support a different set of system names.
Thus each target platform is characterized by a separate xml-file.
These files are created during the build phase of the runtime system
for each target platform.

All possible system elements are located as unordered separate 
elements under the root tag \verb|platform|.
For the individual types of system elements, the following tags are
provided:
\begin{itemize}
\item \verb|<signal ...>| defines a system signal
\item \verb|<interrupt ...>| defines a system interupt
\item \verb|<dation ...>| defines a system dation
\item \verb|<configuration ...>| defines a configuration item
\item \verb|<connection ...>| defines a association provider, if 
   non of the above fit
\end{itemize}

These elements may have attributes 
\begin{description}
\item [name] is the system name
\end{description}

These elements may have other elements as subtree to define parameters, associations and checks.

\paragraph{Note:} The tags are interpreted by the IMC case insensitive. The XML-reader 
expects a  proper XML structure.

\section{Tags in the XML-File}
\subsection{Signal-Tag}
The signal tag contains only the attribute \verb|name| for the
system signal name.

\subsection{Interrupt-Tag}
The interrupt tag contains the attribute \verb|name| for the
system interrupt name.
Possible parameters are listed as subtree \verb|<parameters>|.

\subsection{Dation-Tag}
The dation tag contains the attribute \verb|name| for the
system dation name.
Possible parameters are listed as subtree \verb|<parameters>|.
Possible attributes are listed as subtree \verb|<attributes>|.
Possible associations to this element are
 listed as subtree \verb|<associationProvider>|.


\subsection{Configuration-Tag}
The configuration element has the same structure as the dation tag
except the attribute \verb|name|, which has not to be present.

\paragraph{Note:} A special configuration tag is reserved for
blocking of gpio bits, which are used by system devices like i2c or 
other applications.
\begin{XMLCode}
<configuation name="reservedGpios" autoInstanciate="1">
  <checks>
    <check pinDoesNotCollide="RPiGPIO" bitList="0,1,2,4,6,26"/>
  </checks>
</configuration>
\end{XMLCode}
Regard:
\begin{itemize}
\item autoInstanciate forces one instance to be created automatically.
  This attribute prohibits the usage by the PEARL application in the system part
\item pinDoesNotCollide must have  the same name as used in the gpio rules in 
   e.g. RPiDigitalIn.xml
\item bitList is created from the the configuration item in \textit{menuconfig}.
The tool \texttt{convertList} is used in the \textit{Makefile} to create this list.
\end{itemize}

\subsection{Associations}
The connection between two system elements is treated by the following
elements:

\begin{itemize}
\item \verb|<needAssociation>| with the attribute 
 \verb|name| and the identifier of the expected interface. The 
  interface names are described in the runtime documentation. 
\item \verb|<associationProvider>|  with elements \verb|<associationType>|.
  Attributes for an associationType are:
  \begin{itemize}
   \item  \verb|<name>|,  which is an unique free selectable
     identifier for this kind of  association.
   \item \verb|<clients>|  if given it is the number of allowed
      clients for this connection provider. If not given, there is no
      limits of allowed clients.
   \end{itemize} 
\end{itemize}

\paragraph{Note:} An association may have another association.
 
\section{Parameters}
Only parameters of system elements of types bit, fixed and char are supported.
The actual value may be restricted to definite values and ranges.

Each parameter tag must have an attribute name, which is used as nickname 
for checks of allowed values, combinational useage and error messages.
The name must consist only of characters. Camelcase notation should be used.

Tags are defined for each of them:
\begin{itemize}
\item \verb|BIT| with the attribute \verb|length|, which specifies the 
   allowed length of an actual parameter. \verb|<BIT length="8">...</BIT>|
   specifies a BIT(8) parameter. Shorter actual parameters are extended --- 
   longer parameters induce an error message.
\item \verb|FIXED| with the attribute \verb|length|, which specifies the 
   allowed length of an actual parameter. \verb|<FIXED length="15">...</FIXED>|
   specifies a FIXED(15) parameter. Shorter actual parameters are extended --- 
   longer parameters induce an error message.
\item \verb|CHAR| with the attribute \verb|length|, which specifies the 
   allowed length of an actual parameter. \verb|<CHAR length="15">...</CHAR>|
   specifies a CHAR(15) parameter. Shorter actual parameters are allowed --- 
   longer parameters induce an error message.
\end{itemize}

\subsection{Expressions and Checks for and upon Parameter Values}
Parameters are checked by the IMC against some rules. Each parameter
must provide the attribute name, which  consists only of characters. 

In rules these parameters may be used in expressions.
To access a parameter name, an '\$' must be used as prefix.
Expressions are supported for integer results. The expression 
evaluator treats only the type integer.
Nested expressions must be placed in square brackets:
e.g. \verb|anyText[$start+3]anyOtherText|.


Supported rules:

\begin{tabular}{|l|c|p{8cm}|}
\hline
rule & type & description \\
\hline
VALUES & BIT,FIXED, CHAR &
   a comma separated list of allowed values
   is given. Other values produce an error message. 
No nicknames or expressions are allowed.\\
\hline
ConsistsOf & CHAR &
a comma separated list of allowed values is given. 
The actual parameter must consist of only one or more of these
elements.
No name references or expressions are allowed.\\
\hline
NotEmpty & CHAR & 
   at least one character is required \\   
\hline
FIXEDRANGE & FIXED &
   two comma separated integers or expressions with nicknames
 limits the allowed range
   of actual values --- borders included \\ 
\hline
FIXEDGT & FIXED &
   one integer or expression with nicknames
 limits the allowed range at the low side
   of actual value --- border NOT included \\ 
\hline
\end{tabular}

\paragraph{Note:} If other checks are needed, they must become implemented in 
\verb|checks/CheckSyspart.java|.

\section{Consistency Checks between all System Part Elements}
Different system part elements may violate some hardware restictions.
Thus an GPIO pin may only be used as input or output, or on microcontrollers
an GPIO may only be used as digital io or as alternate function like UART comminication.

The XML-tag \verb|<check...>| defines checks, which are executed
over all defines system elements in all modules.

\subsection{Types of Checks}
%Depending on the concrete system element 
%\begin{itemize}
%\item only one instance is allowed, or
%\item multiple instances are allowed as long as some parameters differ, or
%\item e.g. on I2C devices, the I2C address must be unique / not unique on 
%a dedicated i2c bus.  
%\item GPIO bits a microcontrollers and I2C port expander may only used once, or
%\item a multiplexer channel of a dedicated device may only be used once, or
%\item ...
%\end{itemize}

If such a restriction applies  to a system element some rules must be supplied 
in the platform definition.

To some restrictions must have access to instance parameters and instance association.
Instance parameters are accessed via \$ and their name.

\subsubsection{Number of Instances}
\begin{description}
\item[instances="1"] only one instance of the platform system element
   is allowed (like  Log, Console or StdIn)
\item[instances="oncePerSet" set="..."] defines that only one instance with a 
  specified set of parameters are allowed. To define a set we have access to 
  all parameter names and in case of associations the complete association list.
  The attribute set defines a string identifying the set selector. 
  \newline
  Example: Several I2C busses in a system need different system interfaces

  \begin{PEARLCode}
  SYSTEM;
     ! this sample is ok, different unix interfaces
     i2cbus0 : I2CBus('/dev/i2c-0');
     i2cbus1 : I2CBus('/dev/i2c-1');
  \end{PEARLCode}

  Check definition in platform:
  
  \begin{verbatim}
  <connection name="I2CBus" ...>
    <parameters>
       <CHAR name ="deviceName" length="32767">
          <NotEmpty />
       </CHAR>
    </parameters>
    ...
    <check instances="oncePerSet" set="I2CBus($deviceName)"/>
  \end{verbatim}

  The imc maintains a list of all instances of the platform element.
  Multiple instances with different \textit{deviceNames} are ok.
  Multiple instances with the same \textit{deviceName} result is
  error messages.

  The example enshures that there is only one instanciation of 
  one I2C interface. 
\end{description}

\subsubsection{Pin Usage does not collide}
I/o bits are available on GPIOs of microcontrollers and on port expanders.
The adressing of bits are traditionally done by the parameters
\textit{most significant bit numer} and
\textit{number of bits}. These two parameter define the i/o bits of the 
digital i/o component.

Especially on microcontrollers other system devices use GPIO bits, which 
are only available for ether digital i/o or special function.

\begin{description}
\item[pinDoesNotCollide="..."]

Example: PCF8574-port expander on the I2CBus('/dev/i2c-0')
  \begin{PEARLCode}
  i2cBus: I2CBus('/dev/i2c-0');
  !                             76543210
  ! conflict bitsIn3 use bits     xxx
  !          bitsOut4            xxx
  ! bits 5 and 4 are used duplicate 
  bitsIn3:  PCF8574In('44'B4, 5, 3) --- i2cBus;
  bitsOut4: PCF8574In('44'B4, 6, 3) --- i2cBus;
  \end{PEARLCode}

  Check definition in platform:
  \begin{verbatim}
  <check pinDoesNotCollide="PCF8574($i2cAddr)" start="$start" width="$width"/>  
  \end{verbatim}
  \begin{itemize}
  \item The complete chain of association providers is added automatically
  the pinDoesNotCollide selector. 
  The provider evaluates to \verb|I2CBus('/dev/i2c-0')| even
  the provider is specified by an user supplied name.

  \item \verb|PCF8574| is a fixed text. The same text will be used in the PCF8574In
  device prohibiting the usage of a bit simultaneously as input and output.

  \item \verb|$i2cAddr| evaluates to \verb|'44B4'|

  \item \verb|$start| evaluates to  5 at bitsIn3 

  \item \verb|$width| evaluates to 3

  \end{itemize}

The imc creates a list of used pins on the given instance. 
Note that the numbers are not identical to the pin numbers of
the i/o chip. 
In case a pin is already allocated by another instance an error
message will be emitted.

Note that \$width is optional and defaults to 1.

Note that an optional attribute \verb|errorText=".."| exists, which allows 
to modify the error message. This is useful to create more compreghensive 
error messages if the check is applied on e.g. adc multiplexer channels:
 \texttt{errorText="adc channel"}, which replaces the default text "pin".

\item[i2cAdrDoesNotCollide="..."] adresses the problem that 
  different i2c devices must have different adresses on a dedicated
  i2c bus. E.g. in case of port expander we may have multiple
  instances of the same i2c device on the same bus.

  Example: 

  \begin{PEARLCode}
  SYSTEM;
    ! all ok; bits3 and bits2 use diffenent pins they use
    ! the same i2c device PCF8574.
    i2cbus: I2CBus('dev/i2c-0');
    bits3: PCF8574In('44'B4, 5, 3) --- i2cbus;
    bits2: PCF8574In('44'B4, 7, 2) --- i2cbus;
    temp1: LM75('48') --- i2cbus;
    temp1: LM75('49') --- i2cbus;
  \end{PEARLCode}

  Checks definition in platform:

  \begin{verbatim}
  <dation name="PCF8574In" ..
      <parameters>
      ...
     </parameters>
     <needAssociation name="I2CBusProvider"/>
     <checks>
     <check i2cAdrDoesNotCollide="PCF8574" sharable="$iicAdr" />
     <check pinDoesNotCollide="..." />
     </checks>

   ...
  <dation name="PCF8574Out" ..
      <parameters>
      ...
     </parameters>
     <needAssociation name="I2CBusProvider"/>
     <checks>
        <check i2cAdrDoesNotCollide="PCF8574" sharable="$iicAdr" />
        <check pinDoesNotCollide="..." />
     </checks>
   ...

   <dation name="LM75" ...
     <system name="LM75">
     ...
     <needAssociation name="I2CBusProvider"/>
     ...
     <checks>
     <check i2cAdrDoesNotCollide="LM75" nonSharable="$iicAdr" />
     </checks>

        <connection name="I2CBus">
         <parameters>
            <CHAR name="deviceName" length="32767">
               <NotEmpty />
            </CHAR>
         </parameters>
         <associationProvider>
            <associationType name="I2CBusProvider" />
         </associationProvider>
         <checks>
           <check instances ="oncePerSet" set="I2CBus($deviceName)"/>
         </checks>
      </connection>

  \end{verbatim}

  \paragraph{Note:} The name of the parameter \verb|iicAdr| must not be written
  as \verb|i2cAdr| since a parameter name consists only of letters.

  The imc maintains a list of all I2CBusProvider elements.
  For each I2CBusProvider the imc maintains a list of all  i2c components
  on this i2c bus.
  If the same  adress is used by different devices we have an error condition.
  If the same  adress is used by the same device we have an error condition if
  the device does not support multiple usage like a port expander.
\end{description}

If more checks are required, they can be added in \texttt{imc/checks}. 

\section{Platform File}
Each system part elements is represented by a driver in the OpenPEARL 
source tree and an XML-snipplet. These snipplets are
concatenated to the Platform Definition File during the build  process.

Example:
\begin{XMLCode}
<?xml version="1.0" encoding="UTF-8"?>
<platform file="testPlatform.xml">
      <signal name="FixedRangeSignal"/>
      <signal name="FixedDivideByZeroSignal"/>
      <signal name="FloatIsNaNSignal"/>

      <interrupt name="UnixSignal">
         <parameters>
            <FIXED name="signalNumber" length="31">
               <VALUES>1,2,3,15,16,17</VALUES>
            </FIXED>
         </parameters>
      </interrupt>

      <dation name="Disc">
         <parameters>
            <CHAR name="dir" length="32767"/>	<!-- beliebiger String -->
            <FIXED name="nbrOfFiles" length="31">
               <FIXEDRANGE>1,9999</FIXEDRANGE>
            </FIXED>
         </parameters>
         <attributes>FORWARD, DIRECT, IN, OUT, INOUT, SYSTEM</attributes>
         <data>ALL</data>
         <associationProvider>
            <associationType name="NamedAlphicOutProvider" />
         </associationProvider>
      </dation>

      <dation name="StdIn">
         <parameters>
         </parameters>
         <attributes>
            FORWARD, IN	, SYSTEM
         </attributes>
         <DATA>ALPHIC</DATA>
         <checks>
           <check instances="1"/>
         </checks>
      </dation>

      <dation name="StdOut">
         <parameters>
         </parameters>
	 <attributes>
            FORWARD, OUT, SYSTEM
         </attributes>
         <data>ALPHIC</data>
         <associationProvider>
            <associationType name="AlphicOutProvider" />
         </associationProvider>
         <checks>
           <check instances ="1"/>
         </checks>
      </dation>

      <connection name="I2CBus">
         <parameters>
            <CHAR name="deviceName" length="32767">
               <NotEmpty />
            </CHAR>
         </parameters>
         <associationProvider>
            <associationType name="I2CBusProvider" />
         </associationProvider>
         <checks>
           <check instances ="oncePerSet" set="I2CBus($deviceName)"/>
         </checks>
      </connection>

      <dation name="LM75">
         <parameters>
            <BIT name="iicAdr" length="8">
               <VALUES>'20'B4,'48'B4, '49'B4, '4A'B4, '4B'B4,
                       '4C'B4, '4D'B4 , '4E'B4, '4F'B4</VALUES>
            </BIT>
         </parameters>
	 <attributes>
            BASIC, SYSTEM, IN
         </attributes>
         <data>FIXED(15)</data>
         <needAssociation name="I2CBusProvider"/>
         <checks>
           <check i2cAdrDoesNotCollide="LM75"
                   nonShareable="$iicAdr"/>
         </checks>
      </dation>

      <dation name="PCF8574In">
         <parameters>
            <BIT name="iicAdr" length="8">
               <VALUES>'20'B4, '21'B4, '22'B4, '23'B4,
                       '24'B4, '25'B4 , '26'B4, '27'B4,
                       '38'B4, '39'B4 , '3A'B4, '3B'B4,
                       '3C'B4, '3D'B4 , '3E'B4, '3F'B4
               </VALUES>
            </BIT>
            <FIXED length="7" name="start">
               <FIXEDRANGE>0,7</FIXEDRANGE>
            </FIXED>
            <FIXED length="7" name="width">
               <FIXEDRANGE>1,[$start+1]</FIXEDRANGE>
            </FIXED>
         </parameters>
	 <attributes>
            BASIC, SYSTEM, IN
         </attributes>
         <data>BIT($width)</data>
         <needAssociation name="I2CBusProvider"/>
         <checks>
           <check pinDoesNotCollide="PCF8574xx(I2cAdr=$iicAdr)" 
		start="$start" width= "$width" />
           <check i2cAdrDoesNotCollide="PCF8574"
                   shareable="$iicAdr"/>
         </checks>
      </dation>

      <dation name="PCF8574Out">
         <parameters>
            <BIT name="iicAdr" length="8">
               <VALUES>'20'B4, '21'B4, '22'B4, '23'B4,
                       '24'B4, '25'B4 , '26'B4, '27'B4,
                       '38'B4, '39'B4 , '3A'B4, '3B'B4,
                       '3C'B4, '3D'B4 , '3E'B4, '3F'B4
               </VALUES>
            </BIT>
            <FIXED length="7" name="start">
               <FIXEDRANGE>0,7</FIXEDRANGE>
            </FIXED>
            <FIXED length="7" name="width">
               <FIXEDRANGE>1,[$start+1]</FIXEDRANGE>
            </FIXED>
         </parameters>
	 <attributes>
            BASIC, SYSTEM, OUT
         </attributes>
         <data>BIT($width)</data>
         <needAssociation name="I2CBusProvider"/>
         <checks>
           <check pinDoesNotCollide="PCF8574xx(I2cAdr=$iicAdr)" 
		start="$start" width= "$width"/>
           <check i2cAdrDoesNotCollide="PCF8574"
                   shareable="$iicAdr"/>
         </checks>
      </dation>

  <configuration name="Log">
    <parameters>
      <CHAR name="loglevel" length="4">
         <NotEmpty/>
      </CHAR>
    </parameters>
    <needAssociation name="AlphicOutProvider"/>
    <checks>
       <check instances ="1"/>
    </checks>
  </configuration>

   <connection name="LogFile">
      <parameters>
         <CHAR name="fileName" length="32767">
            <NotEmpty/>
         </CHAR>
      </parameters>
      <associationProvider>
         <associationType name="AlphicOutProvider"/>
      </associationProvider>
      <needAssociation name="NamedAlphicOutProvider"/>
   </connection>

   <dation name="RPiDigitalIn">
      <parameters>
         <FIXED name="start" length="31">
             <FIXEDRANGE>0,27</FIXEDRANGE>
         </FIXED>
         <FIXED length="31" name="width"> <FIXEDGT>0</FIXEDGT></FIXED>
         <CHAR name="pullUpDown" length="1"><VALUES>'u','d','n'</VALUES></CHAR>
      </parameters>
      <attributes> IN, SYSTEM, BASIC</attributes>
         <checks>
           <check pinDoesNotCollide="RPiGPIO" 
		start="$start" width= "$width"/>
           </checks>
      <data>BIT($width)</data>
   </dation>

   <dation name="RPiDigitalOut">
      <parameters>
         <FIXED name="start" length="31">
             <FIXEDRANGE>0,27</FIXEDRANGE>
         </FIXED>
         <FIXED length="31" name="width"> <FIXEDGT>0</FIXEDGT></FIXED>
      </parameters>
         <checks>
           <check pinDoesNotCollide="RPiGPIO" 
		start="$start" width= "$width"/>
           </checks>
      <attributes> OUT, SYSTEM, BASIC</attributes>
      <data>BIT($width)</data>
   </dation>

   <configuration name="reservedGpios" autoInstanciate="1">
     <checks>
      <check pinDoesNotCollide="RPiGPIO" bitList="0,1,2,4,5,6,7"/>
     </checks>
   </configuration>

</platform>
\end{XMLCode}
