\chapter{Introduction}   % 1

%\begin{added}
%This document is derived from the PEARL language report V2.2(?)
%Title page and table of contents were added and the page format 
%is set be the new styles. Documentclass is used instead of documentstyle.
%
%Modified text elements are marked with the new LaTeX commands
%\verb+\addedtext{new text}+ or \verb+\begin{added}new text block\end{added}+.
%Text elements, which are indented to be removed are marked with
%\verb+\removedtext+ or \verb+\begin{removed}...\end{removed}+ respectively.
%Text elements which are not implemeted yet are marked with \verb|tobedone| respectively.
%
%The used version respository supports nothing like the \verb|$ID| in svn.
%Thus the version date must be set manually until an aumatice generation 
%in introduced.
%\end{added}

%%%%%%%%%%%%%%%%%%%%%%%%%%%%%%%%%%%%%%%%%%%%%%%%%

PEARL stands for {\bf P}rocess and {\bf E}xperiment {\bf A}utomation
{\bf R}ealtime {\bf L}anguage; it is a higher programming language that
allows a comfortable, safe and to a large extent computer-independent
programming of multitasking and real-time tasks. PEARL was standardised 
by DIN in different extension steps:

\begin{itemize}
\item DIN 66253, part 1, Basic PEARL, 1981 (subset of Full PEARL)
\item DIN 66253, part 2, Full PEARL, 1982
\item DIN 66253, part 3, Distributed System PEARL, 1988
\end{itemize}

Based on the experiences from hundreds of PEARL projects, a gremium of
PEARL users and implementators worked out a definition of PEARL 90 in
1989 and 1990 (cf. Stieger, K.: ``PEARL 90 --- Die Weiterentwicklung von
PEARL'', in: {\em Informatik-Fachberichte 231}, PEARL '89 Workshop
\"uber Realzeitsysteme, Springer 1989).

PEARL 90 corresponds to Full PEARL, though some language elements not
needed in practice are omitted. On the other hand, PEARL 90 contains
some progressive extensions.

%%%\begin{removed}
%%%The language report presented here describes the language extent of
%%%PEARL 90, first outlining the fundamental characteristics of PEARL 90.
%%%In the following, the language elements will be defined exactly.
%%%\end{removed}

%%%\begin{added}
This report defines the concrete implementation of the PEARL 90
language in the \OpenPEARL{} language, which is a subset of PEARL 90.
Some language elements of 
PEARL 90  are not supported, others are described more detailed.
%%%\end{added}

The appendix contains a list of all data types and their possible
applications, a description of the available predefined functions, a
complete description of the syntax, because for didactical reasons not
all syntactic possibilities are described in the respective sections, as
well as a list of all keywords.

